
%=-=-=-=-=-=-=-=-=-=-=-=-=-=-=-=-=-=-=-=-=-=-=-=-=-=-=
\section{ Discusso dos resultados}
\label{chapter:resultados}
%=-=-=-=-=-=-=-=-=-=-=-=-=-=-=-=-=-=-=-=-=-=-=-=-=-=-=


%\textcolor{red}{ Diante do cenáro atual da loja convém a realização de pequenas, mas eficientes mudanças e adaptações na gestão visual da loja, com o objetivo de diminuir movimentações desnecessárias de produtos e pessoas, além de evitar a interrupção do fluxo de clientes no interior da loja. Tais mudanças irão influenciar diretamente na satisfação dos clientes com a otimização do tempo de entrega, além de reduzir os custos de embarque, desembarque e custos por atrasos. Na análise organizacional desta empresa verificou-se que os produtos além de não estarem bem dispostos, geralmente não estão agrupados por família, ou seja, encontram-se dispersos. }

%\textcolor{red}{Pôde-se identificar também que o depósito não possui nenhuma ferramenta específica de controle de estoques, mas sim um sistema de informação geral da organização, que deveria ser alimentado corretamente, mas que pela falta de atualização de forma precisa, ocorrem falhas no controle de estoque. Segundo o gestor isso ocorre porque a empresa ainda está passando por um processo de mudança em seus sistemas de informatização. }

%\textcolor{red}{O gestor complementou ainda que já teve prejuízos por não conseguir controlar seus estoques corretamente, e que chegou a perder vendas e até clientes.}


%\subsection{Gestão de Estoque}
%\label{subchapter:GestãoEstoque}

%A primeira ação a ser feita na gestão de estoque foi a escolha de um sistema de controle do estoque.  
%A gestão da empresa optou por um sistema próprio, IHL, que tem funções de cadastro, 
%controle de entrada e saída, conseguindo gerar relatórios de vendas diários, semanal, mensal e anual 
%~\ref{fig:telaBuscaProduto}~\ref{fig:telaConsultaPreco}~\ref{fig:telaCompras}. 


%\begin{figure}[htb]
%\centering
%\includegraphics[width=0.3\columnwidth]{Figuras/telaBuscaProduto.jpg}
%\label{fig:telaBuscaProduto}
%\caption{Tela IHS, Busca de produto }
%\end{figure}

%\begin{figure}[htb]
%\centering
%\includegraphics[width=0.3\columnwidth]{Figuras/telaConsultaPreco.png}
%\label{fig:telaConsultaPreco}
%\caption{Tela IHS, Consulta Preço}
%\end{figure}

%\begin{figure}[htb]
%\centering
%\includegraphics[width=0.3\columnwidth]{Figuras/telaCompras.jpg}
%\label{fig:telaCompras}
%\caption{Tela IHS, Compras }
%\end{figure}


%Com os dados em estoque, foi possivel observar as perdas por validade, por quebras e por outros, neste quesito coloca-se, roubo, furto e não conferencia de mercadorias (saída e entrada). Com os relatorios gerados podemos agora efetuar inventarios cicliclos.

%No diagnostico incial, a cada 25 itens verificados, tinha-se variação, nas quantidades em estoque, em 13,5 itens de média. Ocasionando um perda de receita de R\$18,500,00 (em média). Essas perdas, não contabilizadas anteriomente, deixaram de agregar $15\%$ no valor das vendas da empresa. 
%Os pedidos entregues errados aconteciam em $45\%$ das vezes, com o controle de estoque.

%A partir do momento em começamos a fazer o controle de estoque, as perdas no estoque tiveram uma queda, melhorando o FIFO e diminuindo a margem de erros de estoque. Diminui-se para quase zero a existencia de produtos vencidos em estoque. 

%A freqüência com que esses dados estiveram em observação foi quinzenalmente pelo tempo de três meses. 

%Nesta etapa a parte foi colocada em prática o controle de estoque, mas, o projeto que foi desenvolvido para a empresa é um projeto bem amplo que envolve todos os deptos como: Controle de estoques, Pagamentos, Recebimentos, e dentro de todos esses programas contém relatórios detalhados de todas as informações necessárias para se tomar decisões, quer seja para comprar algum item, saber se tem alguma mercadoria que sua venda é muito baixa, entre outras informações.


%\subsection{5S}
%\label{subchapter:5S}

%Com a coleta de dados conseguimos identificas os problemas das empresas e buscou-se a implantação das melhorias.

%O programa 5S foi escolhido para iniciar as melhorias de gestão visual da loja, atacando o layout da loja e reformulando a exibição de produtos. O 5S quando bem implantado, permite que a empresa consiga atingir liberação de espaço físico, melhoria do ambiente de trabalho, maior visualização dos materiais e organização.

%O primeiro senso que utilizamos foi o de limpeza, seguido a de utilização e arrumação. Com a a limpeza do ambiente, achamos avarias em produtos, produtos caidos, e contaminação por poeiras, substancias oleosas. Primeiro passo foi a eliminação das contaminações e limpenza dos produtos do mostruario.

%A seguir fomos definir a utilização dos produtos. No mostruario é adequado manter apenas um item de cada produto para que os consumidores possam verificar o mix de produto que a loja dispoem. O senso de ordenação foi o passo mais desafiador dado que o layout apresenta restrições que demandam estudos para suas soluções. As gondolas na entrada da loja, forma coloados os itens de menor valor e maior saída, conexões hidraulicas ( enroscadas e soldaveis). Indetificados com etiquetas para identificação dos mesmos. 

%Dando continuidade a arrumação da loja, seguimos por distribuir e alocar os materiais em quatro corredores de acordo com familias  a que o item pertence. Um corredor para ferramentas, outro para eletrica, o seguinte para  hidraulica e o quarto corredo para tintas. Com essa organização da loja, o cliente conseguiam percorrer a loja e identificar a possivel localização dos itens que procura. Os sensos de higiene e auto disciplina requerem uma aior concentração de esforços. 

%s colaboradores são bem particpativos e entendem que a melhoria da empresa, representa melhoria no ambiente de trabalha bem como possivel expansão e promoções internas. Os colaboradores receberam uniformes (camisa e bota) e epi’s (protetor auricular, oculos e cinta ortopedica) para uso no ambiente de trabalho.Além disso foram distribuidos kits para os vendedores, que inclui, lápis, marca texto, caneta esferografica, bloco de notas e agenda.

%Após aplicação do programa o monitoramente se manteve para assegurar que as ações implantadas sejam usadas e mantidas. Com o layout da loja, observou-se um crescimento no numero de vendas  de 14% durante a semana e 23% aos fds e feriados. Ainda que a receita proveniente de vendas tenha subido apenas 8%.
