%=-=-=-=-=-=-=-=-=-=-=-=-=-=-=-=-=-=-=-=-=-=-=-=-=-=-=
\section{Referencial Teórico - Definir Tópico}
\label{chapter:secao2}
%=-=-=-=-=-=-=-=-=-=-=-=-=-=-=-=-=-=-=-=-=-=-=-=-=-=-=

\subsection{Gestão de Estoque}
\label{subsection:gestaoEstoque}

\citeonline{Correa2004} nos dizem que, quando as empresas possuem um estoque grande ela pode atender com mais facilidade uma grande quantidade de clientes. No entanto, como saber qual produtor estocar? E a quantidade? 

Determinar essas variáveis (“o que?” e “quanto?” ter em estoque) é sempre uma adversidade enfrentada pelas empresas. Para determinar as demandas de produtos, as empresas, costumam fazer analise do histórico de vendas, afim de estimar quantidade e valor, de vendas futuras. 

Estoques em excesso geram grandes perdas de valor e gasto com armazenamento e mão de obra. Além de que, a mercadoria pode perder parte do seu valor ou perecer ocasionando grande prejuízo para empresa.

Outo fator encontrado em um grande volume de estoque é o espaço para o armazém. Quanto maior o estoque maior será o espaço utilizado pela empresa para guardar esse volume.

O objetivo da gestão de estoque é assegurar um nível adequado de estoque, que seja capaz de sustentar o nível de atividades da empresa ao menor custo. 
\citeonline{Matias2007} observa que os estoques servem para melhorar o atendimento as necessidades da empresa, em um espaço curto de tempo, e a um baixo preço. 

Segundo \citeonline{Correa2004}, várias empresas na década de 1980 tentaram manter estoques mínimos, gerando problemas de abastecimentos e paradas de produção e suspensão de serviços. Formulou-se, na época, a ideia que grande volume em estoque era um desperdício de espaço e dinheiro parado. Está ideia não está totalmente errada, está incompleta. O estoque deve ser aquele que foi estipulado pela empresa.

Trabalhar com estoque mínimo é vantajoso quando os fornecedores de matéria prima se encontram nas proximidades da empresa, e existe confiança nos prazos de entrega. Manter um estoque de segurança é uma maneira que as organizações não criarem custos adicionais. 

O método just in time é um modo de gestão de estoque, onde as empresas buscam diminuir o tempo de fabricação e o tamanho de seus estoques. De origem japonesa, este modelo foi implementado na administração da organização para ajudar na redução de custo para as empresas e seus fornecedores \cite{MAXIMIANO2005}.

Campos (2013) pontuam parâmetros para gerir estoques, que são:
\begin{enumerate}[label=\alph*)]
\item \textbf{Consumo Médio} – É a média aritmética do consumo previsto ou realizado num determinado período.
\item \textbf{Tempo de Reposição} – É o prazo dado desde a emissão de ordens de compra até o atendimento.
\item \textbf{Lote de Encomenda (ou Econômico)} – É a quantidade de material que se compra ou se fabrica de cada vez. Deve-se procurar um tamanho de lote que minimize o custo total anual.
\item \textbf{Estoque de Segurança} – Este é um item delicado que leva em conta a previsão de variação no consumo médio e, também, no tempo de reposição, para equilibrar a reserva de estoque de um lado e os custos de oportunidade de outro.
\item \textbf{Estoque Máximo} - É a quantidade máxima de material a ser mantida em estoque, é a soma do Lote de Encomenda com o Estoque de Segurança. 
\end{enumerate}


\subsubsection{Gestão Visual}
\label{subsubsection:gestaoVisual}

A gestão à vista é uma metodologia ligada à preocupação com a qualidade no trabalho. O programa 5s atua no sentido de permitir saltos de qualidade interna sejam dados por meio da prática da gestão à vista. 

A criação de uma cultura de bons hábitos deve fazer com que os colaboradores envolvidos absorvam as ideias e incorporem-nas como parte constante de sua rotina de trabalho, sempre atentando para os pontos do programa que necessitam de reavaliação e melhorias \cite{ABRANTES2001}. 

\citeonline{GREIF1989} diz que a gestão visual é um poderosa fonte de informação, ajuda a comunicar de forma rápida e eficazmente na empresa, mostrando indices e resultados dos esforços da organização. Comunicação visual é informação self-service – faz a mesma informação comumente disponível e compreensível a todos que a vêem, no exato momento em que a vêem \cite{GREIF1989}. 

A comunicação da informação deixa de estar restringida a um fluxo hierárquico, permitindo que o fluxo se crie por si só. Além disso o fluxo de informação da gestão visual é fundamental num processo de mudança de uma empresa, permitindo uma maior envolvimento de todos os colaboradores. 

Esta não está confinada apenas a quadros de indicadores, imagens instrutivas ou notas de precauções, mas a um conjunto de técnicas que integram a informação nos sistemas operativos, de forma a adicionar valor a cada tarefa produtiva. 

Em suma, a gestão visual aliada a um programa de implementação Lean permite a eliminação dos três tipos de perdas identificados por \citeonline{DREW2004}, uma vez que permite a interpretação rápida e fácil da informação, uma resposta rápida aos problemas e a comunicação entre as equipas de trabalho. Contribui, assim, para uma maior autonomia dos operadores e redução de erros, que resulta numa melhoria do ambiente de trabalho e na unificação da cultura empresarial.



%Na se\c c\~ao de desenvolvimento voc\^es usar\~ao muitas refer\^encias!!!

%\textbackslash nocite\{rotuloDaReferencia\}, faz com que uma cita\c c\~ao que n\~ao foi citada no texto apare\c ca nas Refer%\^encias Bibliogr\'aficas. \textbf{Exemplo de uso:} \textbackslash nocite\{tanebaum2010\} 

%De acordo\citeonline{tanebaum2010}, n\~ao importa o que ele disse. S\'o estou fazendo uma cita\c c\~ao indireta. 

%"($\cdots$) cita\c c\~ao direta com at\'e 3 linhas, cita\c c\~ao direta com at\'e 3 linhas, cita\c c\~ao direta com at\'e 3 linhas, cita\c c\~ao direta com at\'e 3 linhas, cita\c c\~ao direta com at\'e 3 linhas cita\c c\~ao direta com at\'e 3 linhas cita\c c\~ao direta com at\'e 3 linhas ($\cdots$)" \cite[p.~34]{tanebaum2010}.
 
%\vspace{24pt} %dois espacos de 12
%\begin{citacao}
%"Cita\c c\~ao direta com mais de 3 linhas, cita\c c\~ao direta com mais de 3 linhas, cita\c c\~ao direta com mais de 3 linhas, cita\c c\~ao direta com mais de 3 linhas, cita\c c\~ao direta com mais de 3 linhas, cita\c c\~ao direta com mais de 3 linhas, cita\c c\~ao direta com mais de 3 linhas, cita\c c\~ao direta com mais de 3 linhas, cita\c c\~ao direta com mais de 3 linhas, cita\c c\~ao direta com mais de 3 linhas, cita\c c\~ao direta com mais de 3 linhas, cita\c c\~ao direta com mais de 3 linhas, cita\c c\~ao direta com mais de 3 linhas, cita\c c\~ao direta com mais de 3 linhas, cita\c c\~ao direta com mais de 3 linhas. \cite[p.~34]{tanebaum2010}. 
%\end{citacao}
%\vspace{24pt} %dois espacos de 12

%Quando for preencher o arquivo "referencias.bib", no campo "author", se houverem dois ou mais autores, preenche os nomes normalmente separando-os por "and". Exemplo: \textit{Berg Dantas and Juliana Schivani}. Deve-se fazer isso sempre. Caso haja mais de 3 autores, o latex seleciona sozinho o primeiro e completa com o \textit{et al.} (significa "e outros" em latim). 

%\subsection{Materiais}
%Materiais materiais materiais materiais materiais materiais materiais materiais materiais materiais materiais materiais materiais.

%\subsection{M\'etodo}
%M\'etodo m\'etodo m\'etodo m\'etodo m\'etodo m\'etodo m\'etodo m\'etodo m\'etodo m\'etodo m\'etodo m\'etodo m\'etodo.

%\subsubsection{Testes}
%Testes testes testes testes testes testes testes testes testes testes testes testes testes testes testes testes testes.
