
%=-=-=-=-=-=-=-=-=-=-=-=-=-=-=-=-=-=-=-=-=-=-=-=-=-=-=
\section{Metodologia de coleta e análise de dados}
\label{chapter:METODOLOGIA}
%=-=-=-=-=-=-=-=-=-=-=-=-=-=-=-=-=-=-=-=-=-=-=-=-=-=-=

\subsection{Perfil da Empresa}
\label{subsection:perfil_empresa}

\subsection{Coleta de Dados}
\label{subsection:coleta_dados}


\subsection{Análise de Dados}
\label{subsection:analise_dados}






%\textcolor{red}{A pesquisa foi realizada entre dezembro de 2017 a abril de 2018 na loja de materiais de construção, IHS Materiais de Construções.}

%\textcolor{red}{A Empresa IHS Materiais de Construções, constitui-se em uma Empresa de Pequeno Porte situada na cidade de Manaus-AM, no bairro Aleixo, na qual se trabalha com uma gama de produtos essenciais a uma obra, desde materiais básicos (produtos mais pesados, tais como, areia, tijolos, cimentos, ferragens, madeiras, etc.) até produtos para acabamento (produtos mais leves, tais como pincéis, tintas, pias, conexões, entre outros).}

%\textcolor{red}{A loja possui 50 metros de frente por 30 metros de fundo, numa parte da cidade onde o comércio de materiais de construção é bastante intenso. Convém ressaltar que a empresa teve sua fundação no ano de 2000, porém em um tamanho físico menor se comparado ao disponível atualmente, visto que, por ser uma empresa familiar, veio crescendo fisicamente aos poucos, sem ter um planejamento inicial para a construção da mesma, fato que até hoje influencia na organização e disposição dos produtos no interior da loja, os quais, muitas vezes, não estão dispostos da melhor maneira. } 

%\textcolor{red}{Para o desenvolvimento desta pesquisa, o método utilizado foi o estudo de caso, caracterizado como uma investigação empírica de fenômenos contemporâneos no contexto real, em especial quando os limites entre fenômenos e o contexto não são evidentes. }

%\textcolor{red}{Para orientar a da coleta dos dados e aumentar a confiabilidade da pesquisa. \citeonline{YIN2010} sugere o protocolo de estudos de caso, no qual deve conter: visão geral do projeto do estudo de caso, questões do estudo de caso e leituras importantes sobre o tópico a ser investigado, conter as questões específicas para a coleta de dados, planilha para disposição de dados e fontes para responder a cada questão. }

%\textcolor{red}{A análise do estudo de caso, possibilitou a triangulação das fontes de várias evidências permitindo assim a realização de apontamentos positivos e oportunidade de melhoria do processo para a elaboração das conclusões e na e obtenção dos resultados.}
