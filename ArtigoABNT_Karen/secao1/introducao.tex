%=-=-=-=-=-=-=-=-=-=-=-=-=-=-=-=-=-=-=-=-=-=-=-=-=-=-=
\section{Introdução}
\label{chapter:secao1}
%=-=-=-=-=-=-=-=-=-=-=-=-=-=-=-=-=-=-=-=-=-=-=-=-=-=-=

\normalsize
Em tempos de grandes investimentos e crescimento econômico, a busca por diferenciais para garantir aumento das operações e/ou adquirir melhor qualidade de produtos e serviços, sempre foram fatores que instigavam a pesquisa e aplicação de recursos.

Nos dias atuais, com o país saindo de uma severa recessão econômica, essa busca tem aumentado. Os setores que mais buscam inovações são indústria e comercio. Com cada vez mais pessoas buscando seu próprio empreendimento, a competitividade tende a aumentar nos setores de comercio e serviços. Especificamente sobre materiais de construção, no Amazonas encontram-se cerca de $920$ lojas 
~\cite{AMANCO2014}.

O valor total das vendas de materiais de construção no comércio alcançou $R\$119,3$ bilhões em $2016$. Na comparação com o ano anterior houve queda nominal de $4,2$ ~\cite{ABRAMAT2017}.

No comercio as inovações afetam todos os setores dentro do estabelecimento, indo deste as técnicas de vendas, o pós venda e gestão de estoque. Segundo o Sindicato do Comércio Atacadista de Louças, Tintas e Ferragens de Manaus (Simacon), em janeiro de $2017$, havia na cidade de Manaus, 297 lojas de materiais de construção associadas ao sindicato ~\cite{SIMACON2017}.

Em uma loja de materiais de construção, o maior desafio é a gestão de estoque. Evitar que produtos atinjam seu prazo de validade, má conservação, FIFO não efetivo \textcolor{red}{.... (melhorar)}

Segundo, \citeonline{Pascoal2008}, para se manter um estoque organizado, são necessários vários requisitos, uns dos principais são: determinar o número de itens que deve ser estocados, a quantidade necessária para um período determinado e qual o momento certo que deve reabastecer o estoque. E toda a empresa deve ter um estoque mínimo ou também chamado de estoque de segurança, que determina a quantidade mínima estocada, que é destinada caso tenha algum atraso no momento da compra com o fornecedor, garantindo um funcionamento eficiente da empresa, sem riscos de falta ao consumidor.

Desse modo, para que uma empresa tenha um bom desenvolvimento é preciso que tenha uma boa gestão de estoques, controlando e armazenando adequadamente os materiais, para que não haja falhas e posteriormente a falta do produto. Diante disso, o objetivo principal desse trabalho foi analisar a gestão de estoque de uma empresa de materiais de construção localizada no bairro Aleixo na cidade de Manaus. Foi realizado uma pesquisa a partir de observações e um questionário semi-estruturado, para saber pelos próprios funcionários quais seriam os problemas de estoque na empresa em questão. Observaram-se vários problemas de gestão de estoques, como: esquecimentos de materiais pela falta de organização e o principal agravante desta empresa é que todo o processo de contagem de estoques e vendas é feita manualmente, possibilitando assim muitas falhas. 
Desse modo, o trabalho apresentado explicitará quais são os problemas de uma empresa que não tem uma gestão de estoque definida e quais seriam as sugestões possíveis para que se tenha um bom desenvolvimento dentro do estoque e como isso facilitaria a sua gestão.
